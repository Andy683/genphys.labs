\documentclass{../lab_class}

\usepackage{fancyhdr}
\pagestyle{fancy}
\rhead{П.\,Ю. Смирнов, 687 гр.}
\lhead{Лабораторная работа № 4.3.5, МФТИ, весна 2018}

\begin{document}

{\Large 4.3.5 -- Изучение голограммы.}

\paragraph{Цель работы.}
Изучить свойства голограмм точечного источника и объёмного предмета.

В работе используются: гелий-неоновый лазер, голограммы, набор линз, предметная шкала, экран, линейка.

\paragraph{Теоретическая часть.}

\begin{equation}\label{eq:main}
	\rho = \sqrt{m \lambda z_0}
\end{equation}

\paragraph{Эксперимент.}

В работе используется гелий-неоновый лазер с длиной волны $\boxed{\lambda = 532 \ \sn \m}$. Цену деления предметной шкалы определяем по известной формуле для дифракции Фраунгофера $\lambda/D = \Delta x / L$. Имеем расстояние между дифракционными максимумами и расстояние между экраном и кассетой
\begin{gather*}
	\Delta x = 5.5 \ \cm, \\
	L = 118 \ \cm
\end{gather*}
соответственно. Отсюда $\boxed{D = 11.4 \ \smu \m}$.

Теперь осветим лазером голограмму. Дифракционное изображение на экране суть кольца, по радиусы которых (формула \ref{eq:main}) мы можем узнать расстояние до мнимого источника (до источника при записи голограммы при отс. увеличения) $z_0$. Имеем
\begin{equation*}
	\Delta\rho_1 = 1.5 \ \mm, \quad \Delta\rho_2 = 1.5 \ \mm, \quad \Delta\rho_3 = 1.8 \ \mm, \quad \Delta\rho_4 = 2.0 \ \mm,
\end{equation*}
откуда $z_0 \simeq 48.8 \ \m$ -- расстояние до мнимого источника. Расстояние же до источника при записи голограммы, как можно ожидать из воспоминаний о расположении линзы на установке, есть $d \simeq 48.8 \ \cm$.

В следующем опыте мы изучаем фокусирующие свойства самой голограммы, играющей роль короткофокусной линзы. Ранее мы уже нашли цену деления предметной шкалы $D$. Теперь подвинем её вплотную к голограмме. Расстояние от экрана до голограммы $b = 45.5 \ \cm$, наблюдаемое расстояние между штрихами на экране $D' = 2 \ \mm$. Отсюда получаем фокусное расстояние голограммы $f \simeq 45.8 \ \cm = d$. Полученный результат согласуется с другим методом, использованным выше. 

На нашу голограмму записано не абы что, а изображение самой настоящей трёхмерной линейки! Используя способность головы к вращению, получаем, что опорная волна при записи голограммы падала на предмет под углом порядка $\simeq 5^{\circ}$.

\end{document}